\subsection{Token Frequency}
\subsubsection{Motivation}
%TODO does celex have token frequencies?
% TODO citations
% request baayen and lieber
% get bybee in print or request
% have albright 2002, albright and hayes 2003
% have ernestus and baayen
% request hay et al
% from albright's modeling analogy as grammar paper:
%In fact, it appears that the propensity to generalize morphophonological
%patterns to new forms depends primarily on type frequency, and not on token
%frequency. This restriction has been noted numerous times in the literature;
%see Baayen and Lieber (1991) for English derivational suffixes Bybee (1995)
%for French conjugation classes, German past participles, and others, Albright
%(2002) for Italian conjugation classes, Albright and Hayes (2003, p. 133) for
%English past tenses, Ernestus and Baayen (2003, p. 29) for stem-final voicing
%in Dutch, Hay, Pierrehumbert, and Beckman (2004) for medial consonant clusters
%in English, and additional references in Bybee (1995)). 

% TODO lit review
% p.23:
%There is abundant prima facie evidence that the high token frequency of
%diphthongization does not make it a strong pattern: synchronically it is
%relatively unproductive in experimental settings (Bybee and Pardo 1981;
%Albright, Andrade, and Hayes 2001), and diachronically verbs tend to lose
%diphthongization alternations (Penny 2002; Morris 2005). Furthermore,
%overregularization errors among children acquiring Spanish consistently result
%in omitting diphthongization (Clahsen, Aveledo, and Roca 2002), even though
%diphthongizing tokens constitute a large portion—perhaps even the majority—of
%childrens’ experience.

% TODO Albright lit review
% p.23-24:
%In order to test the influence of token frequency more systematically, I ran
%both the GCM and the MGL with and without taking token frequency into account.
%Specifically, a weighting term was introduced in the GCM, so the contribution
%of each analog was defined not only in proportion to its similarity, but also
%in proportion its (log) token frequency. A weighting term was also introduced
%into the MGL, such that the contribution of each word to the hits and/or scope
%of a rule was weighted according to its log token frequency. The result was
%that both models did slightly worse when token frequency was taken into
%account, as shown in (17).

% TODO keep in mind for my methods
% p. 24
%It should be noted that these particular experimental items were not
%constructed for the purpose of dissociating type and token frequency, and
%ultimately the fairest test would be based on items that diverge more in their
%predictions.

%TODO 
% from his gradient phonotactics paper, p. 12:
%None of the models explored here derived any advantage from their ability to take token frequency into account
%➢ GNM and simple NNB models did best when frequency weighting was turned off
%➢ Vitevitch and Luce model, which has frequency weighting built in, does not come out
%ahead because of it
%An apparent difference from Bailey and Hahn (2001), who found significant contribution of token frequency
%➢ However, even there, only a tiny numerical boost was observed (r2 gain of .01?)
%➢ I was unable to replicate this effect, even on their data
%In most cases, taking token frequency into account doesn’t change the predictions at all
%➢ Most words in the lexicon are very low frequency, so a boost for high token frequency only gives more influence to a small set of words
%When it does make a difference, though, it tends to be a deleterious one
%➢ Compared predictions of GNM with and without token frequency, to find those words which changed the most when token frequency was considered
%[zeI] (say), [gEr] (there), [paInt] (mine), l2m (come), [gli:d] (need)
%➢ For all five of these words, the frequency-sensitive model was farther from the human
%ratings (overestimated goodness)
%➢ Further evidence that pattern strength is related to type, not token frequency (Bybee
%1995; Albright 2002b; Albright and Hayes 2003; Hay, Pierrehumbert, and Beckman 2004)

\citet{Albright_gradient_2006,Albright_modeling_2009} has shown that giving
models the ability to use token frequency does not improve their performance,
and can even hinder it. However, the way he incorporated token frequency into
various models made the effect of a sound sequence on the grammar or analogical
system correlate positively with the token frequency of the words those
sequences appeared in.  Thus it is possible that token frequency is
uncorrelated with the productivity of a pattern, but it is also possible that
token frequency is inversely correlated with it.  On one hand, token frequency
may simply not be used to determine grammaticality, as in a model where token
frequency is represented in the lexicon and grammaticality is calculated from a
grammar which only pulls patterns out of the lexicon, abstracting away from
word-specific information like token frequency.  On the other hand, high token
frequency may serve to increase the degree to which the patterns in the word
are associated with that particular word rather than with the grammar in
general. Thus, the patterns found in very frequently used words may be
memorized as exceptional, so that patterns associated with high token
frequencies are less likely to be generalized than others.

% The answer to this question will inform future phonotactic modeling as well as
% shed light on the question of where phonotactic judgments come from. If token
% frequency is irrelevant to judgments, this would support a view in which the
% lexicon does not participate in the formation of judgments directly, only
% indirectly through the formation of the grammar. If token frequency is
% inversely correlated with judgments, this will support a view in which lexical
% information works with the grammar in order to separate exceptions from general
% patterns.

I predict that constraints found mainly in very high token frequency words are not
highly weighted in the phonotactic grammar. This would have implications for how we model
phonotactics, as token frequency would need to be reintroduced as a factor, but as one
whose effect is negative or found by the grammar rather than assumed to be positive. It
would also have implications for theories of how the lexicon interacts with the grammar; token
frequency is closely associated with access to the lexicon, so this finding would support
a view in which the lexicon is involved in phonotactic judgment even if there is a separate
grammar.

\subsubsection{Plan}

The hypothesis is that constraints that hold mostly of high token frequency words,
which we could think of as constraints that encode exceptions, are not highly weighted
in the general phonotactic grammar, the one applied to arbitrary novel words (rather than to exceptional words
and conceivably, to novel words that have specific properties that cause analogy to exceptional words).

This hypothesis predicts that a grammar learned from a lexicon containing highly frequent, exceptional
words will generalize to nonce words less well, that is, less similarly to English speakers, than a grammar
learned from a lexicon with the exceptional words removed.

I will gather a corpus of English words to train a Maximum Entropy grammar on. I will remove
a certain number of words from this corpus several times, resulting in several different corpora
of the same size, and then repeat this process with multiple sizes since there is no \textit{a priori}
reason to believe in a certain threshold on the degree of token frequency that would cause the
effect in question. For each size, one corpus will be formed by removing the most frequent words, and
at least one will be formed by removing a randomly selected set of words.

I will find or elicit judgment data on nonce words that do not seem to cause analogy to
specific English words more than usual. Then, I'll test each learned grammar on these words
and compare their results to the English speaker results. I'll run a regression to see if
the grammars that were trained on lexica with high token frequency words removed systematically
performed better than the ones based on lexica with arbitrary sets of words removed.
% I will use data sets from phonotactic judgment experiments on nonce words
% combined with a corpus of English words that provides token frequencies.  I
% plan to fit four Maximum Entropy models to this data and compare their levels
% of correlation with the data.  First, I will train a model that takes each
% unique word in the data once, ignoring token frequency. Second, I will train a
% model that uses token frequency. Third, I will train a model that uses
% transformed token frequency, so that high frequency words have a smaller rather
% than a greater effect on the model. Finally, I will fit a model on data with
% words above a certain threshold of token frequency removed.
% %TODO i don't know if the last one makes sense

% Based on past research, I predict that the model that ignores token frequency
% will outperform the one that uses it straightforwardly. It is difficult to
% predict the performance of the third and fourth models, but I suspect that one
% or both of them will outperform the model that ignores token frequency, showing
% that high token frequency is not merely irrelevant to phonological learning but
% actually harmful, in the sense that patterns followed by extremely common words
% are likely to be exceptions that should not be generalized.

% I will calculate empirical probabilities of the words based on their token
% frequencies. I will use a phonotactic learner to fit a Maximum Entropy model by
% minimizing KL divergence on these probabilities. Then I will rerun the learner
% with a transformation applied to the empirical distribution so that high token
% frequency words are considered improbable and low token frequency words are
% considered probable.


%Then I will run the Hayes and Wilson learner
%\citep{Hayes2008} on those words two times, once giving the learner each input
%word once per iteration, and once giving the learner each input word a number
%of times proportional to its token frequency each iteration. If the model
%without token frequency performs better than the one with token frequency, I
%will modify the learner to penalize patterns associated with high token
%frequency and see if that one outperforms the one that ignores token frequency.

%TODO could I tweak number of training words to simulate penalty?


