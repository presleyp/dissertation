SurveyMan is a desktop application that communicates with Amazon's Mechanical
Turk survey-running website in order to manage surveys and experiments with
minimal human intervention. A working prototype exists but it has many
shortcomings, particularly for linguistics experiments which often have very
precise requirements. I propose to develop an experiment-specific version of
the software that will meet the needs of our phonotactic and psycholinguistic
web-based studies. This would benefit my own research and that of
any linguist interested in running word or sentence judgment studies on the
internet. Other social scientists would also be likely to find it useful.

%TODO make tables out of prose for exp. materials and maybe also software comparison
%IbexFarm either can or will be able to do audio
%Brian Smith used IbexFarm
%write out about control flow
%map out dependencies
This software would enable researchers without a programming background
to set a wide variety of options on the mechanics and presentation of their
experiments.

There are several existing programs to help researchers run experiments on the web, and they differ
from SurveyMan in various ways.

Survey programs like SurveyMonkey and LimeSurvey are relatively easy to use, but lack some
crucial experiment-centric features. In SurveyMonkey, randomization requires a premium membership,
and neither of them handle Latin square designs for the experimenter.

There are also experiment-running programs developed by scientists, such as \citet{WebExp}, \citet{Experigen},
and \citet{PsiTurk}. These tend to require more programming but have more of the features needed for experiments.
However, they still lack some of the features planned for SurveyMan. WebExp, for instance, lacks support for
acoustic stimuli and constrained randomization of question order. Experigen does not record reaction time. PsiTurk
is compatible with any of these features, but the experimenter has to be able to write JavaScript.

\citet{IbexFarm} is a strong candidate for use in web experiments, as it was designed for linguists and thus addresses
most of the features needed for our experiments. However, SurveyMan has plans for advanced features, such as training
periods that continue until the participant performs well enough to advance.

Thus, SurveyMan will balance the ability to design experiments quickly and easily with minimal programming knowledge
with a wide array of features tailored to the needs of phonologists and other social scientists.
% Many other software packages for web-based experiments \citep{WebExp,Experigen,PsiTurk}
% require researchers to run experiments on their own server, which increases
% barriers to access, but SurveyMan will send experiments directly to Mechanical Turk,
% although the option to use one's own server is also in the works. IbexFarm \citep{IbexFarm} is
% another option that hosts experiments for researchers, but SurveyMan will have features
% that IbexFarm lacks in terms of the ease with which researchers can use resources of
% various types and create complex control flow within their experiments.
% Thus, SurveyMan would increase the accessibility of web-based experiments and
% thus the rate at which experiments can be completed.

% Second, the planned version of SurveyMan will tackle an important outstanding
% question in experimental methods, namely how experimenters ought to determine
% the number of participants they need for an experiment to have sufficient
% power.  The practice of running statistical tests on the data in the middle of
% an experiment and adding participants if the results are not yet significant is
% known to distort results \citep{Armitage1969,Snoop,Simmons2011}, and the
% preferred method of deciding on a sample size is to do a power analysis before
% running the experiment based on the effect size found in previous related
% studies. However, if the effect has not previously been studied, it is
% difficult to choose a sample size appropriately. Some techniques have been
% developed to solve this problem \citep{Todd2001}, %TODO bootstrapping
% and we plan to incorporate them directly into the experimental software so that
% participants are added until the program determines that the distribution has
% stabilized.  This feature has the potential to increase the validity of data
% published in the field.

% Thus, this software could benefit research in phonology and other social sciences
% in terms of both quantity and quality.


% \begin{table}
%     \begin{tabular}[h][ccc]
%         Software & Externally Hosted & Latin Square & Resources & Randomization & Branching & Training Loops & Reaction Times \\
%         SurveyMonkey & \ding{51} & & \ding{51} & \ding{51}* & \ding{51}* &  & \\
%         LimeSurvey & \ding{51} & & \ding{51} & \ding{51} & \ding{51} &  & \\
%         WebExp & &  & \ding{51} & \ding{51} & & & \\
%         Experigen & & & \ding{51} & \ding{51} & & & \\
%         PsiTurk & & \ding{51} & & & & \\
%         IbexFarm & \ding{51} & \ding{51} & \ding{51} & \ding{51} && &\\
%         SurveyMan & \ding{51} & \ding{51} & \ding{51} & \ding{51} & \ding{51} & \ding{51} & \ding{51}\\
%     \end{tabular}
% \end{table}

% *Available in premium version.
% **Allows users to write this themselves but does not provide it.

\subsection{Plan}
In order to make an experimental version of SurveyMan that is fully functional
for word and sentence judgment studies, the following features need to be added:

\begin{itemize}
\item Pseudorandomization as an alternative to pure randomization of question order.
\item Counterbalancing of question types.
% \item Statistics to determine optimal sample size.
% \item Reverse qualifications to keep the same participant from taking the experiment
%     more than once. This has already been implemented.
% \item Ability to show multiple resources (images, sounds, or videos) with a question.
%   This is already possible because questions now take arbitrary HTML.
\item Logging of reaction times and the times of events such as playing sounds or videos.
\item Ability to place restrictions on events; for instance, allowing a sound to be played
only once or allowing a choice to be made only after the sound has played.
\item Training loops.
\item Ability to randomly pair images, audio, and video with questions for each participant.
% \item Ability to change the layout of elements on the page.
\end{itemize}

% Additionally, there are changes we would like to make that are not strictly necessary
% but would make the program more extensible and user-friendly:

% \begin{itemize}
% \item Rewrite the software in Python, both for speed of programming and interoperability
% with R.
% \item Create an R interface for the program to replace the current interface (we are currently
%     starting with a Python interface which will make an R interface possible later).
% \item Use a feedback loop to allow experimenters to interact with pilot studies and
% develop experimental designs iteratively.
% \item Set up a JavaScript library to hold user-submitted functions to extend the functionality
% of the software.
% \end{itemize}

Currently, Emma Tosch is working on the backend of the survey-based version of SurveyMan and Molly McMahon is writing a Python front-end,
which will enable an R interface later on. I will adapt SurveyMan to include the necessary experiment-based
features.

I have implemented all of the features necessary to run my own experiments, and will focus on integrating
those changes with the broader SurveyMan framework before adding additional features.


