The previous section asked the question, given that words with certain
characteristics are preferred to words with other characteristics, how do those
preferences interact? This chapter will back up and investigate which characteristics
create that difference in the first place. I will assume a baseline
of Maximum Entropy phonotactics, where the type frequency of a sound sequence
correlates with weights on constraints referring to it and thus with its
predicted grammaticality in the language.  In addition to type frequency, other
information about a sound sequence may predict its effect on the grammar.  I
will look for effects of the token frequency of the words the sequence appears
in and whether the pattern is eliminated or produced as a result of a
morphological alternation.

\subsection{Alternations}

% never mind, bad motivation for this project
%As (cite Hayes et al, Becker et al) have shown, some sound sequences that are statistically
%over- or under-represented in the lexicon of a language do not affect judgments in
%experiments. In other words, constraints that computational models are likely to
%find and give substantial weight to do not always behave as if they are highly weighted
%for speakers, and the processes they motivate are not always productive. Although inducing
%constraints from lexical statistics is fruitful in many cases, this means that additional
%information is used in the phonological acquisition process. One likely source of information
%is participation of sequences in alternations. Alternations cause morphologically related
%words to differ, and their linkage via meaning and parts of their phonological form may make
%the change, and therefore the sequences that are the inputs and outputs to the change, more
%salient. These sequences may be more likely to be used as constraints and more likely to
%get weights of high magnitude --- negative weights in the case of sequences that alternations
%repair and positive weights in the case of sequences that alternations result in. Alternations
%may give the learner evidence that the over- and under-representation of sequences involved
%in alternations is not merely due to chance and should therefore be attended to. Perhaps other
%constraints are then at a disadvantage relative to these more strongly supported ones. %TODO would that make them less likely to become productive?

% about Gorman
% Gorman isn't talking about what's productive because he didn't do experiments on judgments of
% novel words. He's talking about lexical statistics, somewhat ironically. He's basically just
% saying that the lexical statistics we have show on medial clusters that Pierrehumbert was wrong and that they
% make sense, there are no surprising gaps that we need to account for with some new insight.

%Another factor that has been claimed to affect likelihood of generalization of
%a phonotactic pattern is whether the pattern is the input or output to an
%alternation. Patterns that participate in alternations have direct evidence for
%being preferred or dispreferred, whereas those that are merely under- or
%over-represented in the lexicon have evidence only from lexical statistics.
%\citet{Gorman2013} argues that participation in alternations is key to
%generalization.

Infants show evidence of knowing phonotactic patterns before they begin
producing words \citep{Werker1984,Kuhl2006}, and thus it may be the case that
the phonotactic grammar is learned independently of alternations, which depend
on a lexicon. \citet{Adriaans2010}, for instance, developed a phonotactic
learner based on this assumption. However, it is also possible that infants
have learned about the lexicon of their language before they begin producing
words; furthermore, they may first learn phonotactics in a vacuum but later
incorporate morphological knowledge into their phonotactic grammar. Thus, it is
possible that participation in an alternation improves the salience of a
constraint and causes learners to more heavily weight that constraint, even for
the purposes of phonotactic judgments.

On the other hand, it has not been proven that alternations do increase the
learnability of a phonotactic pattern, and if we can determine that the two
seem to be independent, we could safely model them separately.  Such a finding
would simplify future work on phonotactics and would also have implications for
arguments that the facts of both systems should be derived from the same
machinery. \citet{Pater2003a} gave evidence that alternation learning cannot be
fully modeled without reference to phonotactics, but it is unknown whether
phonotactics can be captured without reference to alternations.

Thus, it is of considerable interest whether the presence of alternations to or
from a sound sequence affects judgments of the grammaticality of that sequence,
holding the sequence's type frequency in the lexicon constant.

It is difficult to find a case in natural language with these properties in
order to do a well-controlled test of the idea, so I will use an artificial
language learning experiment to test the effect of alterations elsewhere in a
language on identical words.

%Ideas I want to hit on here:
%\begin{itemize}
%\item \citet{Adriaans2010} model of online, lexicon-free phonotactics --- is he right about the problem statement?
%\item \citet{Pater2003a} work on whether phonotactics affects alternations --- does it go both ways?
%\item Models of phonotactics --- can they be developed independently from models of alternations?
%\item Modularity --- are these separate modules?
%\item Grammar vs.~lexicon --- if phonotactics don't need to refer to morphology, this supports a grammar
    %abstracted away from the lexicon
%\end{itemize}

The results of this study will inform future modeling as well as theories about
the interaction between phonotactics and alternations.



\subsubsection{Plan}
%I will modify the Hayes and Wilson learner to reward patterns that are the
%outputs of alternations and penalize those that are the inputs to alternations.
%I will compare the performance of this modified learner to that of the original
%on data from previous phonotactic judgment studies, in order to see whether
%participation in alternations is a good predictor of the weight of a constraint
%in a phonotactic grammar.

I will address this question with an experiment in which participants are
trained in an artificial language and then asked for judgments about the probability
that novel words could belong to the artificial language.

Two constraints will be constructed. They will be chosen to be of similar
levels of attestion in English, segmental length, featural specificity, and
phonetic naturalness, but perfect control is not necessary due to the
experimental design. It is, however, necessary that neither be inviolable in
English, or else the experiment will get ceiling effects. For purposes of exposition
I will call these constraints *AB and *CD. At least initially, I plan for these constraints
to actually be *bf and *nf, where the rules are b $\rightarrow$ p / \_f and n $\rightarrow$ m / \_f.
The rules will apply across syllable boundaries, where these rules are not obligatory in English.
However, even heterosyllabic instances of these bigrams are uncommon or unattested in English,
so I will analyze the results for ceiling effects. If needed, I will switch to completely non-English constraints,
such as vowel harmony and consonant harmony.

Two artificial languages will be constructed. The lexicons of the two languages
will be as similar as possible, and neither language will allow sequences of AB
or CD. However, Language 1 will use B as a plural ending, causing alternations
from AB to XB, while Language 2 will use D as a plural ending, causing
analogous alternations from CD to YD.  Thus, in Language 1 AB is the
alternating sequence, and in Language 2 CD is the alternating sequence.

Participants will be shown words and definitions from their assigned language
until they have learned the meanings. They will then be asked to give judgments
on novel words containing AB and CD.

\ex. Languages
    \a. Both Languages:\\
    No bf\\
    No nf\\
    plural: -fa
    \b. Language I: b $\rightarrow$ p / \_f
    \c. Language II: n $\rightarrow$ m / \_f

\ex. Example training words
    \a. Language I: 
dela, delafa, vem, vemfa, sirab, sirapfa, \ldots
    \b. Language II:
dela, delafa, vem, vemfa, siran, siramfa, \ldots

\ex. Example testing words for both languages\\
malo, kamfin, labfu, \ldots

% Language I: *AB, *CD, A $\rightarrow$ X / \_B

% Language II: *AB, *CD, C $\rightarrow$ Y / \_D


% Language I rule: b $\rightarrow$ p / \_f

% Language II rule: n $\rightarrow$ m / \_f

% Language I training: singulars and plurals with meanings

% dela, delafa, vem, vemfa, sirab, sirapfa, \ldots

% Language II training: singulars and plurals with meanings

% dela, delafa, vem, vemfa, siran, siramfa, \ldots

% Testing: singulars

% malo, kamfin, labfu, \ldots


The most straightforward judgment task for this experiment would be asking participants
to simply answer the yes/no question ``Is this word possible in the language you learned?''
Thus, in the first run of the experiment, I will ask participants this question for words
containing AB, CD, XB, and YD, and filler words. However, there is some concern
that when asked yes/no questions, participants try to balance the number of yeses and nos
they give. I will instruct them not to do this, but I will also analyze the results of the
experiment to look for such behavior. If I suspect that such a strategy has been used, I
will rerun the experiment with a two-alternative forced choice task in which I compare
words with AB to words with XB and words with CD to words with YD.

The results will be analyzed with a mixed effects model predicting proportion
of times the banned bigrams are chosen based on the interaction of language learned
and constraint tested, with random effects for subject and item.
If there is a significant effect of whether a sequence was an
alternating sequence on the proportion of times the sequence was chosen in the
forced choice task, I will conclude that evidence from alternations are used in
the phonotactic grammar.
I will look separately at whether the sequence that the rule eliminates is
especially dispreferred and at whether the sequence that the rule creates is
especially preferred, as these may yield different answers.

% it would also be interesting to do this where each language violated the constraint in
% verbs. do exceptions have a different effect on alternating sequences than on non?

